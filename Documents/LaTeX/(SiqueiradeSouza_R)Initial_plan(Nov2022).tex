%%%%%%%%%%%%%%%%%%%%%%%%%%%%%%%%%%%%%%%%%%%%%%%%%%%%%%%%%%%%%%%%%%%%%%%%%%%%%%%%
%2345678901234567890123456789012345678901234567890123456789012345678901234567890
%        1         2         3         4         5         6         7         8
%% memo.tex
%% V1.1
%% 2017/10/203
%% by Prof. Rui Santos Cruz
%% based on a template created by Rob Oakes
%%%%%%%%%%%%%%%%%%%%%%%%%%%%%%%%%%%%%%%%%%%%%%%%%%%%%%%%%%%%%%%%%%%%%%%%%%%%%%%%
\documentclass[a4paper,12pt]{texMemo}
% --> Please Choose the MAIN LANGUAGE for the document in package BABEL
% --> by replacing "main=" in language name selector. Default is "main=english"
\usepackage{hyperref}
\usepackage[english]{babel} % Defines Main language
\usepackage[utf8]{inputenc}
\usepackage{iflang}
\usepackage{ifthen}
\usepackage{parskip}
%\setlength{\parindent}{15pt}

%%%%%%%%%%%%%%%%%%%%%%%%%%%%%%%%%%%%%%%%%%%%%%%%%%%%%%%%%%%%%%%%%%%%%%%%%%%%%%%%
%	MEMO INFORMATION --> Write your info in the following tags.
%%%%%%%%%%%%%%%%%%%%%%%%%%%%%%%%%%%%%%%%%%%%%%%%%%%%%%%%%%%%%%%%%%%%%%%%%%%%%%%%
\memofrom{Dr Rodrigo Siqueira, Paulo Chagas and Dr Diego Szczupak} % Sender(s) Name
\memoid{Axonal Guidance} % Student ID
\memocourse{University of Sheffield, UFRJ, University of Pittsburgh} % Course Name, or abbreviated acronym
\memosubject{6 months from start date} % Subject
\memodate{\today} % Date, -> set to \today for automatically print todays date
%%%%%%%%%%%%%%%%%%%%%%%%%%%%%%%%%%%%%%%%%%%%%%%%%%%%%%%%%%%%%%%%%%%%%%%%%%%%%%%%


\begin{document}
	

\includegraphics[height=0.8in]{Figures/Sheffield.png}   
\hfill 
\includegraphics[height=0.8in]{Figures/Pittsburgh.png}  


\vspace{1cm}

\maketitle % Print the memo header information



%%%%%%%%%%%%%%%%%%%%%%%%%%%%%%%%%%%%%%%%%%%%%%%%%%%%%%%%%%%%%%%%%%%%%%%%%%%%%%%%
%	MEMO CONTENT --> Your content is written here
%%%%%%%%%%%%%%%%%%%%%%%%%%%%%%%%%%%%%%%%%%%%%%%%%%%%%%%%%%%%%%%%%%%%%%%%%%%%%%%%

\section{Key planning points:}

\subsection{NetLogo ou R?}

Netlogo 3d parece funcionar, mas o Netlogo tem limitações de performance e de tempo do programador. \\ \\
R tem o rgl, que faz os gráficos. O lance é só que tem que reconstruir a roda de visualização, e nem sei se rola de reconstruir a roda de visualização a cada tick.

\subsection{Start simple}

\begin{itemize}

\item 3D
\item uma região, bilateral, LM;
\item regras pro axônio;
\item ver se formam feixes, com essas regras;

\end{itemize}

\pagebreak

\subsection{Full model}

\begin{itemize}
	
\item desenho do que seria uma simplificação do córtex;
\item demarcação de regiões;
\item a cada tick, cria um axônio em alguma região, aleatoriamente;
\item os axônios saem andando em random walk, mas com preferência de ir na direção da LM;
\item se chega na LM, sinaliza que cruzou
\item continua andando aleatoriamente até chegar em outra região ou (se for a regra) em outro axônio
\item se chegar em outra região, para ali
\item se chegar em outro axônio, fascicula e passa a seguir ele
\item se andar por muito tempo e não chegar, morre
\item roda até ter um número de axônios ou até ficar cheio/muitos morrendo
\item faz a histologia virtual, conta onde eles chegaram e tal

\end{itemize}

\vspace{\fill}
\begin{center}
	\rule{74ex}{.1mm}\\
	\textbf{Active Touch Laboratory, University of Sheffield}\\
	Department of Psychology, Sheffield S1 2LT, UK\\ 
	\href{http://activetou.ch}{http://activetou.ch}
	
\end{center}

\normalsize


%%%%%%%%%%%%%%%%%%%%%%%%%%%%%%%%%%%%%%%%%%%%%%%%%%%%%%%%%%%%%%%%%%%%%%%%%%%%%%%%
\end{document}